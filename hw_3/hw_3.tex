\documentclass[]{article}
\usepackage{lmodern}
\usepackage{amssymb,amsmath}
\usepackage{ifxetex,ifluatex}
\usepackage{fixltx2e} % provides \textsubscript
\ifnum 0\ifxetex 1\fi\ifluatex 1\fi=0 % if pdftex
  \usepackage[T1]{fontenc}
  \usepackage[utf8]{inputenc}
\else % if luatex or xelatex
  \ifxetex
    \usepackage{mathspec}
  \else
    \usepackage{fontspec}
  \fi
  \defaultfontfeatures{Ligatures=TeX,Scale=MatchLowercase}
\fi
% use upquote if available, for straight quotes in verbatim environments
\IfFileExists{upquote.sty}{\usepackage{upquote}}{}
% use microtype if available
\IfFileExists{microtype.sty}{%
\usepackage{microtype}
\UseMicrotypeSet[protrusion]{basicmath} % disable protrusion for tt fonts
}{}
\usepackage[margin=1in]{geometry}
\usepackage{hyperref}
\hypersetup{unicode=true,
            pdfborder={0 0 0},
            breaklinks=true}
\urlstyle{same}  % don't use monospace font for urls
\usepackage{color}
\usepackage{fancyvrb}
\newcommand{\VerbBar}{|}
\newcommand{\VERB}{\Verb[commandchars=\\\{\}]}
\DefineVerbatimEnvironment{Highlighting}{Verbatim}{commandchars=\\\{\}}
% Add ',fontsize=\small' for more characters per line
\usepackage{framed}
\definecolor{shadecolor}{RGB}{248,248,248}
\newenvironment{Shaded}{\begin{snugshade}}{\end{snugshade}}
\newcommand{\KeywordTok}[1]{\textcolor[rgb]{0.13,0.29,0.53}{\textbf{{#1}}}}
\newcommand{\DataTypeTok}[1]{\textcolor[rgb]{0.13,0.29,0.53}{{#1}}}
\newcommand{\DecValTok}[1]{\textcolor[rgb]{0.00,0.00,0.81}{{#1}}}
\newcommand{\BaseNTok}[1]{\textcolor[rgb]{0.00,0.00,0.81}{{#1}}}
\newcommand{\FloatTok}[1]{\textcolor[rgb]{0.00,0.00,0.81}{{#1}}}
\newcommand{\ConstantTok}[1]{\textcolor[rgb]{0.00,0.00,0.00}{{#1}}}
\newcommand{\CharTok}[1]{\textcolor[rgb]{0.31,0.60,0.02}{{#1}}}
\newcommand{\SpecialCharTok}[1]{\textcolor[rgb]{0.00,0.00,0.00}{{#1}}}
\newcommand{\StringTok}[1]{\textcolor[rgb]{0.31,0.60,0.02}{{#1}}}
\newcommand{\VerbatimStringTok}[1]{\textcolor[rgb]{0.31,0.60,0.02}{{#1}}}
\newcommand{\SpecialStringTok}[1]{\textcolor[rgb]{0.31,0.60,0.02}{{#1}}}
\newcommand{\ImportTok}[1]{{#1}}
\newcommand{\CommentTok}[1]{\textcolor[rgb]{0.56,0.35,0.01}{\textit{{#1}}}}
\newcommand{\DocumentationTok}[1]{\textcolor[rgb]{0.56,0.35,0.01}{\textbf{\textit{{#1}}}}}
\newcommand{\AnnotationTok}[1]{\textcolor[rgb]{0.56,0.35,0.01}{\textbf{\textit{{#1}}}}}
\newcommand{\CommentVarTok}[1]{\textcolor[rgb]{0.56,0.35,0.01}{\textbf{\textit{{#1}}}}}
\newcommand{\OtherTok}[1]{\textcolor[rgb]{0.56,0.35,0.01}{{#1}}}
\newcommand{\FunctionTok}[1]{\textcolor[rgb]{0.00,0.00,0.00}{{#1}}}
\newcommand{\VariableTok}[1]{\textcolor[rgb]{0.00,0.00,0.00}{{#1}}}
\newcommand{\ControlFlowTok}[1]{\textcolor[rgb]{0.13,0.29,0.53}{\textbf{{#1}}}}
\newcommand{\OperatorTok}[1]{\textcolor[rgb]{0.81,0.36,0.00}{\textbf{{#1}}}}
\newcommand{\BuiltInTok}[1]{{#1}}
\newcommand{\ExtensionTok}[1]{{#1}}
\newcommand{\PreprocessorTok}[1]{\textcolor[rgb]{0.56,0.35,0.01}{\textit{{#1}}}}
\newcommand{\AttributeTok}[1]{\textcolor[rgb]{0.77,0.63,0.00}{{#1}}}
\newcommand{\RegionMarkerTok}[1]{{#1}}
\newcommand{\InformationTok}[1]{\textcolor[rgb]{0.56,0.35,0.01}{\textbf{\textit{{#1}}}}}
\newcommand{\WarningTok}[1]{\textcolor[rgb]{0.56,0.35,0.01}{\textbf{\textit{{#1}}}}}
\newcommand{\AlertTok}[1]{\textcolor[rgb]{0.94,0.16,0.16}{{#1}}}
\newcommand{\ErrorTok}[1]{\textcolor[rgb]{0.64,0.00,0.00}{\textbf{{#1}}}}
\newcommand{\NormalTok}[1]{{#1}}
\usepackage{longtable,booktabs}
\usepackage{graphicx,grffile}
\makeatletter
\def\maxwidth{\ifdim\Gin@nat@width>\linewidth\linewidth\else\Gin@nat@width\fi}
\def\maxheight{\ifdim\Gin@nat@height>\textheight\textheight\else\Gin@nat@height\fi}
\makeatother
% Scale images if necessary, so that they will not overflow the page
% margins by default, and it is still possible to overwrite the defaults
% using explicit options in \includegraphics[width, height, ...]{}
\setkeys{Gin}{width=\maxwidth,height=\maxheight,keepaspectratio}
\IfFileExists{parskip.sty}{%
\usepackage{parskip}
}{% else
\setlength{\parindent}{0pt}
\setlength{\parskip}{6pt plus 2pt minus 1pt}
}
\setlength{\emergencystretch}{3em}  % prevent overfull lines
\providecommand{\tightlist}{%
  \setlength{\itemsep}{0pt}\setlength{\parskip}{0pt}}
\setcounter{secnumdepth}{0}
% Redefines (sub)paragraphs to behave more like sections
\ifx\paragraph\undefined\else
\let\oldparagraph\paragraph
\renewcommand{\paragraph}[1]{\oldparagraph{#1}\mbox{}}
\fi
\ifx\subparagraph\undefined\else
\let\oldsubparagraph\subparagraph
\renewcommand{\subparagraph}[1]{\oldsubparagraph{#1}\mbox{}}
\fi

%%% Use protect on footnotes to avoid problems with footnotes in titles
\let\rmarkdownfootnote\footnote%
\def\footnote{\protect\rmarkdownfootnote}

%%% Change title format to be more compact
\usepackage{titling}

% Create subtitle command for use in maketitle
\newcommand{\subtitle}[1]{
  \posttitle{
    \begin{center}\large#1\end{center}
    }
}

\setlength{\droptitle}{-2em}
  \title{}
  \pretitle{\vspace{\droptitle}}
  \posttitle{}
  \author{}
  \preauthor{}\postauthor{}
  \date{}
  \predate{}\postdate{}


\begin{document}

\begin{longtable}[]{@{}l@{}}
\toprule
title: `ggplot2 application (car:: Salaries)'\tabularnewline
author: ``cafepeng''\tabularnewline
date: ``2018年3月18日''\tabularnewline
output: html\_document\tabularnewline
helpful link:
{[}\url{https://support.zendesk.com/hc/en-us/articles/203691016-Formatting-text-with-Markdown\#topic_xqx_mvc_43__row_tf4_bmn_1n}{]}\tabularnewline
\bottomrule
\end{longtable}

\textbf{Dataset:}

\begin{quote}
The 2008-09 nine-month academic salary for Assistant Professors,
Associate Professors and Professors in a college in the U.S. The data
were collected as part of the on-going effort of the college's
administration to monitor salary differences between male and female
faculty members.
\end{quote}

\begin{itemize}
\tightlist
\item
  Variables:

  \begin{itemize}
  \tightlist
  \item
    rank: AssocProf, AsstProf, Prof
  \item
    discipline:a factor with levels A (`theoretical')or B (`applied'
    departments)
  \item
    yrs.since.phd: years since PhD
  \item
    yrs.service: years of service
  \item
    sex: a factor with levels Female Male
  \item
    salary: nine-month salary, in dollars
  \end{itemize}
\end{itemize}

\begin{enumerate}
\def\labelenumi{\arabic{enumi}.}
\tightlist
\item
  pull the dataset ``Salaries'' from `car' package, and check the
  structure and so on.
\end{enumerate}

\begin{Shaded}
\begin{Highlighting}[]
\KeywordTok{library}\NormalTok{(car)}
\NormalTok{data01<-}\KeywordTok{data.frame}\NormalTok{(Salaries)}
\KeywordTok{str}\NormalTok{(data01)}
\end{Highlighting}
\end{Shaded}

\begin{verbatim}
## 'data.frame':    397 obs. of  6 variables:
##  $ rank         : Factor w/ 3 levels "AsstProf","AssocProf",..: 3 3 1 3 3 2 3 3 3 3 ...
##  $ discipline   : Factor w/ 2 levels "A","B": 2 2 2 2 2 2 2 2 2 2 ...
##  $ yrs.since.phd: int  19 20 4 45 40 6 30 45 21 18 ...
##  $ yrs.service  : int  18 16 3 39 41 6 23 45 20 18 ...
##  $ sex          : Factor w/ 2 levels "Female","Male": 2 2 2 2 2 2 2 2 2 1 ...
##  $ salary       : int  139750 173200 79750 115000 141500 97000 175000 147765 119250 129000 ...
\end{verbatim}

\begin{Shaded}
\begin{Highlighting}[]
\KeywordTok{dim}\NormalTok{(data01)}
\end{Highlighting}
\end{Shaded}

\begin{verbatim}
## [1] 397   6
\end{verbatim}

\begin{Shaded}
\begin{Highlighting}[]
\KeywordTok{summary}\NormalTok{(data01)}
\end{Highlighting}
\end{Shaded}

\begin{verbatim}
##         rank     discipline yrs.since.phd    yrs.service        sex     
##  AsstProf : 67   A:181      Min.   : 1.00   Min.   : 0.00   Female: 39  
##  AssocProf: 64   B:216      1st Qu.:12.00   1st Qu.: 7.00   Male  :358  
##  Prof     :266              Median :21.00   Median :16.00               
##                             Mean   :22.31   Mean   :17.61               
##                             3rd Qu.:32.00   3rd Qu.:27.00               
##                             Max.   :56.00   Max.   :60.00               
##      salary      
##  Min.   : 57800  
##  1st Qu.: 91000  
##  Median :107300  
##  Mean   :113706  
##  3rd Qu.:134185  
##  Max.   :231545
\end{verbatim}

\begin{Shaded}
\begin{Highlighting}[]
\CommentTok{#especially check the variables I am interested in}
\KeywordTok{summary}\NormalTok{(data01$rank)}
\end{Highlighting}
\end{Shaded}

\begin{verbatim}
##  AsstProf AssocProf      Prof 
##        67        64       266
\end{verbatim}

\begin{Shaded}
\begin{Highlighting}[]
\KeywordTok{summary}\NormalTok{(data01$salary)}
\end{Highlighting}
\end{Shaded}

\begin{verbatim}
##    Min. 1st Qu.  Median    Mean 3rd Qu.    Max. 
##   57800   91000  107300  113706  134185  231545
\end{verbatim}

\begin{Shaded}
\begin{Highlighting}[]
\KeywordTok{summary}\NormalTok{(data01$sex)}
\end{Highlighting}
\end{Shaded}

\begin{verbatim}
## Female   Male 
##     39    358
\end{verbatim}

\begin{itemize}
\tightlist
\item
  \textbf{Q1. If income increases as the year of service increases? }
\end{itemize}

\begin{Shaded}
\begin{Highlighting}[]
\CommentTok{#Load the packages needed for data visualization (note: eval=FALSE prevents code from being evaluated)}
\KeywordTok{library}\NormalTok{(ggplot2)}
\CommentTok{#note: alpha refers to the degree of transparancy so that the points would not block other points}
\NormalTok{pic1<-}\KeywordTok{ggplot}\NormalTok{(data01)+}\KeywordTok{geom_point}\NormalTok{(}\DataTypeTok{mapping=}\KeywordTok{aes}\NormalTok{(}\DataTypeTok{x=}\NormalTok{yrs.service,}\DataTypeTok{y=}\NormalTok{salary),}\DataTypeTok{alpha=}\DecValTok{1}\NormalTok{/}\DecValTok{2}\NormalTok{,}\DataTypeTok{color=}\StringTok{"blue"}\NormalTok{,}\DataTypeTok{size=}\DecValTok{5}\NormalTok{)}
\NormalTok{pic1}
\end{Highlighting}
\end{Shaded}

\includegraphics{hw_3_files/figure-latex/unnamed-chunk-2-1.pdf}

\begin{itemize}
\item
  \textbf{Q1 Finding: as years of service increases, the salaries
  increase as well}
\item
  \textbf{Q2.is there a difference in salaries between males and females
  overall?}
\end{itemize}

\begin{Shaded}
\begin{Highlighting}[]
\CommentTok{# add another aesthetic measure "sex"}
\KeywordTok{library}\NormalTok{(ggplot2)}
\NormalTok{pic2<-}\KeywordTok{ggplot}\NormalTok{(data01)+}\KeywordTok{geom_point}\NormalTok{(}\DataTypeTok{mapping=}\KeywordTok{aes}\NormalTok{(}\DataTypeTok{x=}\NormalTok{yrs.service,}\DataTypeTok{y=}\NormalTok{salary,}\DataTypeTok{color=}\NormalTok{sex),}\DataTypeTok{alpha=}\DecValTok{1}\NormalTok{/}\DecValTok{2}\NormalTok{,}\DataTypeTok{size=}\DecValTok{5}\NormalTok{)}
\NormalTok{pic2}
\end{Highlighting}
\end{Shaded}

\includegraphics{hw_3_files/figure-latex/unnamed-chunk-3-1.pdf}

\begin{itemize}
\item
  \textbf{Q2 Findings: it seems that the salary range is relatively
  limited to females}
\item
  \textbf{Q3\_1. following Q2, is it possible that the difference btw
  males and females gets larger in higher income range(e.g.~higher than
  Q3)?}
\end{itemize}

\begin{Shaded}
\begin{Highlighting}[]
\KeywordTok{library}\NormalTok{(ggplot2)}
\NormalTok{pic3_1<-}\KeywordTok{ggplot}\NormalTok{(data01)+}\KeywordTok{geom_point}\NormalTok{(}\DataTypeTok{mapping=}\KeywordTok{aes}\NormalTok{(}\DataTypeTok{x=}\NormalTok{yrs.service,}\DataTypeTok{y=}\NormalTok{salary>}\DecValTok{134185}\NormalTok{,}\DataTypeTok{color=}\NormalTok{sex),}\DataTypeTok{alpha=}\DecValTok{1}\NormalTok{/}\DecValTok{2}\NormalTok{,}\DataTypeTok{size=}\DecValTok{5}\NormalTok{)}
\NormalTok{pic3_1}
\end{Highlighting}
\end{Shaded}

\includegraphics{hw_3_files/figure-latex/unnamed-chunk-4-1.pdf}

\begin{itemize}
\item
  \textbf{Q3-1 Finding: due to the relative small sample of female, this
  picture is relatively meaningless}
\item
  \textbf{Q3\_2.Use ``facet''" setting to draw two diagrams that
  represents females and males seperately}
\end{itemize}

\begin{Shaded}
\begin{Highlighting}[]
\KeywordTok{library}\NormalTok{(ggplot2)}
\NormalTok{pic3_2<-}\KeywordTok{ggplot}\NormalTok{(data01)+}\KeywordTok{geom_point}\NormalTok{(}\DataTypeTok{mapping=}\KeywordTok{aes}\NormalTok{(}\DataTypeTok{x=}\NormalTok{yrs.service,}\DataTypeTok{y=}\NormalTok{salary),}\DataTypeTok{color=}\StringTok{"blue"}\NormalTok{,}\DataTypeTok{alpha=}\DecValTok{1}\NormalTok{/}\DecValTok{2}\NormalTok{,}\DataTypeTok{size=}\DecValTok{5}\NormalTok{)+}\KeywordTok{facet_wrap}\NormalTok{(~sex)}
\NormalTok{pic3_2}
\end{Highlighting}
\end{Shaded}

\includegraphics{hw_3_files/figure-latex/unnamed-chunk-5-1.pdf}

\begin{itemize}
\item
  \textbf{Q3\_2 Finding: females' salary range tends to be relatively
  limited compared to males}
\item
  \textbf{4\_1.Does the salary differs in terms of ranks?}
\end{itemize}

\begin{Shaded}
\begin{Highlighting}[]
\KeywordTok{library}\NormalTok{(ggplot2)}
\NormalTok{pic4_1<-}\KeywordTok{ggplot}\NormalTok{(data01,}\DataTypeTok{mapping=}\KeywordTok{aes}\NormalTok{(}\DataTypeTok{x=}\NormalTok{yrs.service,}\DataTypeTok{y=}\NormalTok{salary))+}
\StringTok{  }\KeywordTok{geom_point}\NormalTok{(}
      \DataTypeTok{mapping=}\KeywordTok{aes}\NormalTok{(}\DataTypeTok{color=}\NormalTok{rank),}\DataTypeTok{alpha=}\DecValTok{1}\NormalTok{/}\DecValTok{2}\NormalTok{,}\DataTypeTok{size=}\DecValTok{5}\NormalTok{)+}
\StringTok{  }\KeywordTok{geom_smooth}\NormalTok{(}
      \DataTypeTok{mapping=}\KeywordTok{aes}\NormalTok{(}\DataTypeTok{color=}\NormalTok{rank))}
\NormalTok{pic4_1}
\end{Highlighting}
\end{Shaded}

\begin{verbatim}
## `geom_smooth()` using method = 'loess'
\end{verbatim}

\includegraphics{hw_3_files/figure-latex/unnamed-chunk-6-1.pdf}

\begin{itemize}
\tightlist
\item
  \textbf{Q4\_1 Finding: higher the rank, higher the variance in
  salaries}
\end{itemize}

\emph{Note: we can use the command `ggsave' to save the diagram.
\texttt{ggsave(filename="pic4\_1.pdf",plot=pic4\_1)}}

\begin{itemize}
\tightlist
\item
  \textbf{Q4\_2: Take a closer look at the highest rank ``professor'' by
  plotting its line only}
\end{itemize}

\begin{Shaded}
\begin{Highlighting}[]
\KeywordTok{library}\NormalTok{(ggplot2)}
\KeywordTok{require}\NormalTok{(dplyr) }\CommentTok{# filter is part of dplyr package}
\end{Highlighting}
\end{Shaded}

\begin{verbatim}
## Loading required package: dplyr
\end{verbatim}

\begin{verbatim}
## 
## Attaching package: 'dplyr'
\end{verbatim}

\begin{verbatim}
## The following object is masked from 'package:car':
## 
##     recode
\end{verbatim}

\begin{verbatim}
## The following objects are masked from 'package:stats':
## 
##     filter, lag
\end{verbatim}

\begin{verbatim}
## The following objects are masked from 'package:base':
## 
##     intersect, setdiff, setequal, union
\end{verbatim}

\begin{Shaded}
\begin{Highlighting}[]
\NormalTok{pic4_2<-}\KeywordTok{ggplot}\NormalTok{(data01,}\DataTypeTok{mapping=}\KeywordTok{aes}\NormalTok{(}\DataTypeTok{x=}\NormalTok{yrs.service,}\DataTypeTok{y=}\NormalTok{salary))+}\KeywordTok{geom_point}\NormalTok{(}\DataTypeTok{mapping=}\KeywordTok{aes}\NormalTok{(}\DataTypeTok{color=}\NormalTok{rank),}\DataTypeTok{alpha=}\DecValTok{1}\NormalTok{/}\DecValTok{2}\NormalTok{,}\DataTypeTok{size=}\DecValTok{5}\NormalTok{)+}\KeywordTok{geom_smooth}\NormalTok{(}\DataTypeTok{data=}\KeywordTok{filter}\NormalTok{(data01,rank==}\StringTok{"Prof"}\NormalTok{), }\DataTypeTok{se=}\OtherTok{FALSE}\NormalTok{)}
\NormalTok{pic4_2}
\end{Highlighting}
\end{Shaded}

\begin{verbatim}
## `geom_smooth()` using method = 'loess'
\end{verbatim}

\includegraphics{hw_3_files/figure-latex/unnamed-chunk-7-1.pdf}

\textbf{Q4\_2 Finding: higher the rank, highe the range in salaries}


\end{document}
